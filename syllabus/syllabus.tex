\documentclass[10pt]{article}

\usepackage[letterpaper, portrait, margin=1.25in]{geometry}

\usepackage{authblk}
\usepackage[yyyymmdd,hhmmss]{datetime}
\usepackage{fancyhdr}
\pagestyle{fancy}
\lhead{ASTR 324: Introduction to AstroStatistics and Big Data in Astronomy}
\rhead{Spring 2018}
\rfoot{\em \tiny Compiled on \today\ at \currenttime}
\cfoot{\thepage}
\lfoot{}


%%%%%%%%%%%%%%%%%%%%%%%%%%%%%%%%%%%%%%%%%%%%%%%%%%%%
%%% author-defined commands
\newcommand\about     {\hbox{$\sim$}}
\newcommand\x         {\hbox{$\times$}}
\newcommand\othername {\hbox{$\dots$}}
\def\eq#1{\begin{equation} #1 \end{equation}}
\def\eqarray#1{\begin{eqnarray} #1 \end{eqnarray}}
\def\eqarraylet#1{\begin{mathletters}\begin{eqnarray} #1 %
                  \end{eqnarray}\end{mathletters}}
\def\non    {\nonumber \\}
\def\DS     {\displaystyle}
\def\E#1{\hbox{$10^{#1}$}}
\def\sub#1{_{\rm #1}}
\def\case#1/#2{\hbox{$\frac{#1}{#2}$}}
\def\about  {\hbox{$\sim$}}
\def\x      {\hbox{$\times$}}
\def\ug               {\hbox{$u-g$}}
\def\gr               {\hbox{$g-r$}}
\def\ri               {\hbox{$r-i$}}
\def\iz               {\hbox{$i-z$}}
\def\a                {\hbox{$a^*$}}
\def\O                {\hbox{$O$}}
\def\E                {\hbox{$E$}}
\def\Oa               {\hbox{$O_a$}}
\def\Ea               {\hbox{$E_a$}}
\def\Jg               {\hbox{$J_g$}}
\def\Fg               {\hbox{$F_g$}}
\def\J                {\hbox{$J$}}
\def\F                {\hbox{$F$}}
\def\N                {\hbox{$N$}}
\def\dd               {\hbox{deg/day}}
\def\mic              {\hbox{$\mu{\rm m}$}}
\def\Mo{\hbox{$M_{\odot}$}}
\def\Lo{\hbox{$L_{\odot}$}}
\def\comm#1           {\tt #1}
\def\refto#1          {\ref #1}
\def\T#1              {({\bf #1})}
\def\H#1              {({\it #1})}

%%%%%%%%%%%%%%%%%%%%%%%%%%%%%%%%%%%%%%%%%%%%%%%%%%%%


\title{ASTR 324: Introduction to AstroStatistics and Big Data in Astronomy}
\author{\v{Z}eljko Ivezi\'{c}  \& Mario Juri\'{c}}
\affil{University of Washington, Spring Quarter 2018}
\date{\vspace{-5ex}}

\begin{document}
\maketitle

\vskip 0.3in
\leftline{{\bf  Location and Time:} Tuesdays and Thursdays 2:30-3:50, {\bf B360}}

\vskip 0.2in
\leftline{{\bf  Office Hours:} { Any time when our office doors are open; }}
\leftline{\hskip 1.15in After Tue class, or Tue and Thu mornings are the best.}
\vskip 0.2in
\leftline{{\bf  Grading:} 10 homeworks, 8\% each; final exam: 20\%}
\leftline{                 \hskip 0.8in key: $>$90\%=A, $>$80\%=B, $>$70\%=C, $>$50\%=D.}
\vskip 0.2in
\leftline{{\bf  Class web site:} https://github.com/uw-astr-324-s18/astr-324-s18}
\vskip 0.2in
\leftline{\bf  Reference textbook:} 
\leftline{Ivezi\'{c}, Connolly, VanderPlas \& Gray: {\it Statistics, Data Mining, and Machine Learning in Astronomy: }}
\leftline{{\it A Practical Python Guide for the Analysis of Survey Data} (Princeton University Press, 2014)}
\leftline{See http://press.princeton.edu/titles/10159.html}
\vskip 0.3in

{\bf Learning Goals:}

This course will introduce students to most common statistical and computer science methods 
used in astronomy and other physical sciences. It will combine theoretical background with 
examples of data analysis based on modern astronomical datasets. Practical data analysis 
will be done using python tools, with emphasis on astroML module (see www.astroML.org). 
While focused on astronomy, this course should be useful to all students interested in data 
analysis in physical sciences and engineering. The lectures will be aimed at undergraduate 
students and the main discussion topics will be based on  Chapters 4 and 5, and selected 
topics from Chapters 6-10, from the reference textbook. 

By taking this course, students will develop basic understanding of topics such as robust 
statistics, hypothesis testing, maximum likelihood analysis, Bayesian statistics, model 
parameter estimation, the goodness of fit and model selection, density estimation and 
clustering, unsupervised and supervised classification, dimensionality reduction, 
regression and time series analysis. Most of these topics will be applied in class homeworks 
to analysis of astronomical data. 

{\bf Prerequisites:}
The students taking this class are required to have basic calculus and basic python skills, 
as well as basic scientific measurements and statistics skills at the level of a freshman lab. 

{\bf Lecture format:}
New material will typically be covered during the first class in a week, while
the second class in a week will be more focused on practical data analysis work. 

\newpage 
% \vskip 0.2in
\leftline{\bf  Class Schedule:}
\begin{itemize}
\item WEEK 1 (starting Mar 27): 
      Introduction to class: syllabus, literature, astroML, python, matplotlib, GitHub. 
      
\item WEEK  2 (starting Apr 3): Introduction to statistics (probability, distributions, 
             robust statistics, Central Limit Theorem,  hypothesis testing).

\item WEEK  3 (starting Apr 10):  Maximum likelihood and applications in astronomy.

\item WEEK  4 (starting Apr 17):  Bayesian statistics and introduction to MCMC.

\item WEEK  5 (starting Apr 24):  Model parameter estimation and model selection.

\item WEEK  6 (starting May 1):   Time series analysis.

\item WEEK  7 (starting May 8):   Big data in astronomy.

\item WEEK  8 (starting May 15): Dimensionality reduction and regression.

\item WEEK  9 (starting May 22):  Density estimation and clustering.

\item WEEK  10 (starting May 29):  Unsupervised and supervised classification.

\item FINAL EXAM: June 6 (Tue, 2:20pm, A210): cake and closed book final exam.

\end{itemize}


\vskip 0.2in

\leftline{\bf  Homework}

There will be ten homeworks, assigned on Thursdays, and due next Thursday. They will
be centered on practical work using python and designed to test the weekly progress. 

We will use modern software engineering tools, GitHub and Jupyter notebooks for HW submission. 

\end{document}

